\documentclass[10pt]{article}

% DO NOT EDIT THE LINES BETWEEN THE TWO LONG HORIZONTAL LINES

%---------------------------------------------------------------------------------------------------------

% Packages add extra functionality.
\usepackage{times,graphicx,epstopdf,fancyhdr,amsfonts,amsthm,amsmath,algorithm,algorithmic,xspace,hyperref}
\usepackage[left=1in,top=1in,right=1in,bottom=1in]{geometry}
\usepackage{sect sty}	%For centering section headings
\usepackage{enumerate}	%Allows more labeling options for enumerate environments 
\usepackage{epsfig}
\usepackage[space]{grffile}

% This will set LaTeX to look for figures in the same directory as the .tex file
\graphicspath{.} %The dot means current directory.

\pagestyle{fancy}

\lhead{\WilliamsID}
\chead{Problem Set \PSNumber \ --- Problem \ProblemNumber}
\rhead{\today}
\lfoot{CSci 256: Algorithm Design}
\cfoot{\thepage}
\rfoot{Spring 2018}

% Some commands for changing header and footer format
\renewcommand{\headrulewidth}{0.4pt}
\renewcommand{\headwidth}{\textwidth}
\renewcommand{\footrulewidth}{0.4pt}

% These let you use common environments
\newtheorem{claim}{Claim}
\newtheorem{definition}{Definition}
\newtheorem{theorem}{Theorem}
\newtheorem{lemma}{Lemma}
\newtheorem{observation}{Observation}
\newtheorem{question}{Question}

\setlength{\parindent}{0cm}


%---------------------------------------------------------------------------------------------------------

% DON'T CHANGE ANYTHING ABOVE HERE

% Edit below as instructed
\newcommand{\WilliamsID}{W1573209}	% Put you Williams ID in the braces
\newcommand{\PSNumber}{7}			% Put the problem set # in the braces
\newcommand{\ProblemNumber}{1}		% Put the problem # in the braces
\newcommand{\ProblemHeader}{Problem \ProblemNumber}	% Don't change this

\begin{document}

\vspace{\baselineskip}	% Add some vertical space

\vspace{\baselineskip}	% Add some vertical space
\textbf{Solution}

To solve this problem, first we want to show that Hitting Set is in NP. We do this by showing that there is some way to verify a proposed solution to this problem in polynomial time. Given a subset H, we can quickly check if it's the right size and if each element in $B_i$ is in $H$. So Hitting Set is in NP. \newline
Next, Hitting Set looks like a covering problem so we will use Vertex Cover and show that $Vertex Cover <_p Hitting Set$. To do this, we will transform an instance of Vertex Cover $(G,k)$ into a version of Hitting Set $((A, B_1,...,B_n), k)$. So graph $G$ is some vertex set and some edge set. To translate this to hitting set, we can imagine the vertices of $G$ as our set $A$, and each edge as representing a subset $B_i$ such that each two vertices connected by an edge is a subset $B_i$ of $A$. With this construction, we can see that a subset of A that hits every subset $B_i$ is exactly a subset of vertices that touches every edge. Therefore, we have a hitting set of size $k$ precisely when we have a vertex cover of size $k$. Additionally, if we do not have a vertex cover of size $k$ then we do not have a hitting set of size $k$. 

% DO NOT DELETE ANYTHING BELOW THIS LINE
\end{document}
