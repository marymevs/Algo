\documentclass[10pt]{article}

% DO NOT EDIT THE LINES BETWEEN THE TWO LONG HORIZONTAL LINES

%---------------------------------------------------------------------------------------------------------

% Packages add extra functionality.
\usepackage{times,graphicx,epstopdf,fancyhdr,amsfonts,amsthm,amsmath,algorithm,algorithmic,xspace,hyperref}
\usepackage[left=1in,top=1in,right=1in,bottom=1in]{geometry}
\usepackage{sect sty}	%For centering section headings
\usepackage{enumerate}	%Allows more labeling options for enumerate environments 
\usepackage{epsfig}
\usepackage[space]{grffile}

% This will set LaTeX to look for figures in the same directory as the .tex file
\graphicspath{.} %The dot means current directory.

\pagestyle{fancy}

\lhead{\WilliamsID}
\chead{Problem Set \PSNumber \ --- Problem \ProblemNumber}
\rhead{\today}
\lfoot{CSci 256: Algorithm Design}
\cfoot{\thepage}
\rfoot{Spring 2018}

% Some commands for changing header and footer format
\renewcommand{\headrulewidth}{0.4pt}
\renewcommand{\headwidth}{\textwidth}
\renewcommand{\footrulewidth}{0.4pt}

% These let you use common environments
\newtheorem{claim}{Claim}
\newtheorem{definition}{Definition}
\newtheorem{theorem}{Theorem}
\newtheorem{lemma}{Lemma}
\newtheorem{observation}{Observation}
\newtheorem{question}{Question}

\setlength{\parindent}{0cm}


%---------------------------------------------------------------------------------------------------------

% DON'T CHANGE ANYTHING ABOVE HERE

% Edit below as instructed
\newcommand{\WilliamsID}{W1573209}	% Put you Williams ID in the braces
\newcommand{\PSNumber}{7}			% Put the problem set # in the braces
\newcommand{\ProblemNumber}{3}		% Put the problem # in the braces
\newcommand{\ProblemHeader}{Problem \ProblemNumber}	% Don't change this

\begin{document}


\vspace{\baselineskip}	% Add some vertical space

\vspace{\baselineskip}	% Add some vertical space
\textbf{Solution}

Given a proposed solution to MI Sched, we can check that it is of size $k$ and that no two intervals overlap, so we have shown that this problem is in NP. \newline
We want to produce a disjoint set of elements so we will use Independent Set and show that $Indp. Set <_p M.I.Sched$. To do this, we will transform an instance of Independent Set $(G,k)$ into an instance of M.I.Sched $((n, O_1,...O_i), k)$ where $n$ is the number of jobs that we have each $O_i$ is a pair of jobs that happen at the same time. Given our graph $G$ we imagine each vertex in $G$ as a job $n$, and each edge in $G$ connecting jobs that overlap. So our answer to M.I.Sched becomes a set of vertices in $G$ that do not share an edge. When we have an independent set in $G$ of size $k$ we have a solution to M.I.Sched of size $k$ and when we do not have such an independent set, we do not have a solution to M.I.Sched. 

% DO NOT DELETE ANYTHING BELOW THIS LINE
\end{document}
