\documentclass[10pt]{article}

% DO NOT EDIT THE LINES BETWEEN THE TWO LONG HORIZONTAL LINES

%---------------------------------------------------------------------------------------------------------

% Packages add extra functionality.
\usepackage{times,graphicx,epstopdf,fancyhdr,amsfonts,amsthm,amsmath,algorithm,algorithmic,xspace,hyperref}
\usepackage[left=1in,top=1in,right=1in,bottom=1in]{geometry}
\usepackage{sect sty}	%For centering section headings
\usepackage{enumerate}	%Allows more labeling options for enumerate environments 
\usepackage{epsfig}
\usepackage[space]{grffile}

% This will set LaTeX to look for figures in the same directory as the .tex file
\graphicspath{.} %The dot means current directory.

\pagestyle{fancy}

\lhead{\WilliamsID}
\chead{Problem Set \PSNumber \ --- Problem \ProblemNumber}
\rhead{\today}
\lfoot{CSci 256: Algorithm Design}
\cfoot{\thepage}
\rfoot{Spring 2018}

% Some commands for changing header and footer format
\renewcommand{\headrulewidth}{0.4pt}
\renewcommand{\headwidth}{\textwidth}
\renewcommand{\footrulewidth}{0.4pt}

% These let you use common environments
\newtheorem{claim}{Claim}
\newtheorem{definition}{Definition}
\newtheorem{theorem}{Theorem}
\newtheorem{lemma}{Lemma}
\newtheorem{observation}{Observation}
\newtheorem{question}{Question}

\setlength{\parindent}{0cm}


%---------------------------------------------------------------------------------------------------------

% DON'T CHANGE ANYTHING ABOVE HERE

% Edit below as instructed
\newcommand{\WilliamsID}{W1573209}	% Put you Williams ID in the braces
\newcommand{\PSNumber}{7}			% Put the problem set # in the braces
\newcommand{\ProblemNumber}{4}		% Put the problem # in the braces
\newcommand{\ProblemHeader}{Problem \ProblemNumber}	% Don't change this

\begin{document}


\vspace{\baselineskip}	% Add some vertical space

\vspace{\baselineskip}	% Add some vertical space
\textbf{Solution}

To solve this problem, we first want to show that the Zero Weight Cycle Problem (ZWCP) is in NP. Given a proposed solution to this problem, we can easily verify that it is in fact a simple cycle whose sum of its weights are zero, so we have shown that this problem is in NP. \newline
We can use subset sum to show that the ZWCP is NP complete. Given an instance of subset sum $(S_1,...S_n, w)$ we can turn this into a graph G using the following construction: We can construct a graph with $n + 1$ nodes, where there is a node 0 and a node for each $S_i$. We draw edges from nodes $S_i$ to $S_j$ and we give this edge the weight $S_i$. We draw an edge from each $S_i$ back to 0 and give it the weight $S_i$, and we draw an edge from node 0 to each $S_i$ and give each of these edges the weight $-w$. Now, a solution to subset sum is represented as a cycle from node 0 to each node $S_i$ in the solution, back to 0. We see that if we have a correct solution to subset sum, this cycle must have a weight 0 because the weight that is $-w$ cancels out the weights that summed up to $w$ in the cycle. On the other hand, we have a 0 weight cycle, then the weights of nodes in the cycle must have summed up to $w$ because we had to use an edge with weight $-w$ to complete the cycle. 
% DO NOT DELETE ANYTHING BELOW THIS LINE
\end{document}
