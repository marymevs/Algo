\documentclass[10pt]{article}

% DO NOT EDIT THE LINES BETWEEN THE TWO LONG HORIZONTAL LINES

%---------------------------------------------------------------------------------------------------------

% Packages add extra functionality.
\usepackage{times,graphicx,epstopdf,fancyhdr,amsfonts,amsthm,amsmath,algorithm,algorithmic,xspace,hyperref}
\usepackage[left=1in,top=1in,right=1in,bottom=1in]{geometry}
\usepackage{sect sty}	%For centering section headings
\usepackage{enumerate}	%Allows more labeling options for enumerate environments 
\usepackage{epsfig}
\usepackage[space]{grffile}

% This will set LaTeX to look for figures in the same directory as the .tex file
\graphicspath{.} %The dot means current directory.

\pagestyle{fancy}

\lhead{\WilliamsID}
\chead{Problem Set \PSNumber \ --- Problem \ProblemNumber}
\rhead{\today}
\lfoot{CSci 256: Algorithm Design}
\cfoot{\thepage}
\rfoot{Spring 2018}

% Some commands for changing header and footer format
\renewcommand{\headrulewidth}{0.4pt}
\renewcommand{\headwidth}{\textwidth}
\renewcommand{\footrulewidth}{0.4pt}

% These let you use common environments
\newtheorem{claim}{Claim}
\newtheorem{definition}{Definition}
\newtheorem{theorem}{Theorem}
\newtheorem{lemma}{Lemma}
\newtheorem{observation}{Observation}
\newtheorem{question}{Question}

\setlength{\parindent}{0cm}


%---------------------------------------------------------------------------------------------------------

% DON'T CHANGE ANYTHING ABOVE HERE

% Edit below as instructed
\newcommand{\WilliamsID}{W1573209}	% Put you Williams ID in the braces
\newcommand{\PSNumber}{7}			% Put the problem set # in the braces
\newcommand{\ProblemNumber}{2}		% Put the problem # in the braces
\newcommand{\ProblemHeader}{Problem \ProblemNumber}	% Don't change this

\begin{document}


\vspace{\baselineskip}	% Add some vertical space

\vspace{\baselineskip}	% Add some vertical space
\textbf{Solution}

To solve this problem, we first want to show that Mono Sat is in NP. If given a solution to this problem, we can verify its correctness by checking that the solution is of size $k$ and then seeing if it satisfies the expression. So Mono Sat is in NP. \newline
Though this problem mentions satisfiability, it appears to be a covering problem, so we will use Vertex Cover to show that $Vertex Cover <_p Mono Sat$. Given an instance of Vertex Cover $(G,k)$, we want to transform it into an instance of Mono Sat $((T, C), k)$, where $T$ is the set of terms and $C$ is the set of clauses. Graph $G$ is a vertex set and an edge set. We can imagine this graph to represent Mono Sat where each vertex is a term $\in T$ and each edge joins terms that are in the same clause $C$. A solution to Mono Sat then becomes a vertex cover for our graph $G$ where we want to set the terms in our vertex cover to 1 to satisfy the expression. So we can see that if we have a Vertex Cover of size $k$ then we have a solution to Mono Sat of size $k$ and if there is no such vertex cover, then we do not have a solution. 

% DO NOT DELETE ANYTHING BELOW THIS LINE
\end{document}
